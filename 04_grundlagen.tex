\section{Theoretische Grundlagen}
\label{sec:grundlagen}
%zugrundeliegende Theorien und Modelle
% Definitionen
% Stand der Technik, Normen und Standards, Wirtschaftliche Aspekte
% persönliche Positionierung

\subsection{Verdickermittel}

\subsection{Strömung und Förderdruck}

\paragraph{Strömungsform:}
Um den Dosierstrom entsprechend seiner Strömungseigenschaften charakterisieren zu können, wird zunächst mit der sogennanten \textsc{Reynoldszahl} die Strömungsform bestimmt. Sie ist eine dimensionslose Kennzahl und beschreibt das Verhältnis zwischen Tragheitskräften zu Reibungskräften in strömenden Flüssigkeiten und ist für durchströmte Rohrleitungen unter Gleichung \eqref{eq: reynolds} definiert. %\cite{Römpp} 

\begin{equation}
	\label{eq: reynolds}
	Re = \frac{d_H*\rho*\overline{u}}{\eta}
\end{equation}
\begin{parameter}
	Re 			& 	\textsc{Reynoldszahl} \\
	\eta 		& dynamische Viskosität des Fluids\\
	\rho 		& Dichte des Fluids\\
	d			&	hydraulischer Rohrdurchmesser\\
	\overline{u} & mittlere Strömungsgeschwindigkeit\\
\end{parameter}

Abhängig von bestimmten Grenzwerten, den sogennanten kritischen Reynoldszahlen, lässt sich nun die Strömungsform zuordnen. 


\subsection{\textsc{Nikuradse}-\textsc{Colebrook}-\textsc{Moody}-Diagramm}
\subsection{Gesetz von  \textsc{Hagen}-\textsc{Poiseuille}}


\subsection{Stand der Technik}
\subsection{Dosierpumpen}
\subsubsection{oszillierende Verdrängerpumpen}
\subsubsection{rotierende Verdrängerpumpen}

\subsection{Normen und Standards}

\subsection{Rotationsviskosimeter}
\subsection{Densimeter}

Norm für Viskositätsbestimmung

https://www.din.de/de/neuer-inhalt/wdc-beuth:din21:306904236
https://www.din.de/de/wdc-beuth:din21:512291
https://www.din.de/de/neuer-inhalt/wdc-beuth:din21:329765890

\subsection{Wirtschaftliche Aspekte und Entwicklungsperspektiven}