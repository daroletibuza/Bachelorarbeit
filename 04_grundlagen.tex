\newpage
\section{Theoretische Grundlagen}
\label{sec:grundlagen}
%zugrundeliegende Theorien und Modelle
% Definitionen
% Stand der Technik, Normen und Standards, Wirtschaftliche Aspekte
% persönliche Positionierung

\subsection{Verdickungsmittel}

In der Farben- und Putzindustrie werden Verdickungsmittel als rheologische Additive bezeichnet. Sie erhöhen die Viskosität von Flüssigkeiten und ändern somit ihre rheologischen Eigenschaften, welche die Auftragungs-, Fließ- und Verlaufseigenschaften von Farben und Putzen bestimmen. Verdickungsmittel kommen jedoch auch in der Lebensmittelchemie oder Pharmazie zum Einsatz. Je nach dem welche Anforderungen an das Verdickermittel gestellt werden, unterscheiden sich diese in ihrer Zusammensetzung (siehe Abb. \ref{fig:verdicker_einteilung}). \cite{Brock.2009}

\begin{figure}[h!]
	\centering
	\begin{forest}
		forked edges,
		for tree={draw,align=center,edge={-latex}}
		[Verdickungsmittel, for children={fit=band}
			[Anorganisch]
			[Organisch, for children={fit=band}
				[Niedermolekular]
				[Synthetisch]
				[Natürlich, for children={fit=band}
					[Natürlich (abgewandelt)]
					]	
			]	
		]
	\end{forest}	
	\caption{Einteilung von Verdickungsmitteln nach Zusammensetzung \cite{Brock.2009}}
	\label{fig:verdicker_einteilung}
\end{figure}
\FloatBarrier

Je nach Verdickungsmittel können verschiedene Effekte wie Gelbildung, Solvatation, Ausbildung von Netzstrukturen, Coulomb-Kräfte, Quellung und Wasserstoff- Brückenbindungen, sowie deren gegenseitige Einflussnahme die Erhöhung der Zähflüssigkeit bewirken. \cite{Brock.2009} \\
Betrachtet man speziell die sogenannten assoziativen Verdickungsmittel lassen sich diese den Gruppen der abgewandelten, natürlichen Verdicker und den synthetischen Verdickern zuordnen.  Sie spezifizieren sich gegenüber anderen Verdickertypen darin, dass sie neben hydrophilen Gruppen auch hydrophobe End- und Seitengruppen enthalten, welche dem Verdickungsmittel einen Tensidcharakter verleihen. Deshalb bestehen assoziative Verdicker unteranderem aus hydrophob modifizierten Polymerstrukturen (siehe Abb. \ref{fig:assoziativ_einteilung}).

\begin{figure}[h!]
	\centering
	\begin{forest}
		forked edges,
		for tree={draw,align=center,edge={-latex}}
		[Assoziativverdicker
		[hydrophob \\ modifizierte \\ Polyacrylate]
		[hydrophob \\ modifizierte \\ Celluloseether]
		[hydrophob \\ modifizierte \\ Polyether]
		[hydrophob \\ modifizierte \\ Polyacrylamide]
		[assoziative \\ Polyurethan-Verdicker]
		]
	\end{forest}	
	\caption{Einteilung von Assoziativ-Verdickern nach chemischer Struktur \cite{Brock.2009}}
	\label{fig:assoziativ_einteilung}
\end{figure}
\FloatBarrier
Diese strukturelle Eigenschaft der Assoziativ-Verdicker macht die Bildung von Micellen möglich und es treten neben der Quellung in der Wasserphase sogenannte "`Micellbrücken"' zwischen Latex-Teilchen der Bindemitteldisperion auf, welche zusätzlich Viskositätserhöhung bewirken. \cite{Brock.2009} \pagebreak

In der geplanten Dosierung soll auf das Verdickungsmittel TAFIGEL PUR 85 der \mbox{\textsc{Münzing Chemie Gmbh}} zurückgegriffen werden. Laut Hersteller handelt es sich hierbei um einen assoziativen Polyurethan-Verdicker, welcher durch Gerüstbildung zwischen Verdickermolekülen, Bindemittel und Pigmentpartikeln die gewünschte Viskosität hervorruft und stabilisiert. Diese Beschreibung deckt sich mit den vorangegangenen Beschreibung der Assoziativverdicker.\cite{MunzingChemieGmbH.2014}\\
In Abbildung \ref{fig:struktur_puverdicker} ist eine schematische Struktur eines solchen assoziativen Polyurethan-Verdickers aufgeführt. Diese beispielhafte Struktur zeigt hydrophile, höher molekulare Polyethersegmente, welche über Urethan-Gruppen verbunden sind und durch hydrophobe Molekülgruppen verknüpft werden. \cite{Brock.2009}

\begin{figure}[h!]
	\centering
	\includegraphics[width=1.0\textwidth]{img/verdicker_struktur}
	\caption{schematische Struktur eines assoziativen Polyurethan-Verdickers, \linebreak erstellt nach \cite{Brock.2009}}
	\label{fig:struktur_puverdicker}
\end{figure}
\FloatBarrier
%Ende

Durch diesen Mix der hydrophoben und hydrophilen Strukturen wird der Tensidcharakter des Verdickungsmittels bestimmt und es ergeben sich Netzstrukturen mit assoziierten "`Micellbrücken"', wie in Abbildung \ref{fig: verdicker_anwendung} dargestellt. Zusätzlich ist zu erkennen, dass auch Wechselwirkungen mit bereits vorhandenen Tensidmolekülen in der Dispersion auftreten können und die Struktur somit weiter stabilisieren. \cite{Mezger.2016}

\begin{figure}[h!]
	\centering
	\includegraphics[width=1.0\textwidth]{img/verdicker_anwendung}
	\caption{Netzstruktur durch Verdickermittel in Latex-Dispersion (mit und ohne Tensid), \linebreak erstellt nach \cite{Mezger.2016}}
	\label{fig: verdicker_anwendung}
\end{figure}
\FloatBarrier
%Ende

Aufgrund dieser ausgeprägten Netzstrukturen innerhalb des Verdickungsmittels ist es jedoch auch möglich, dass das Verdickungsmittel selbst eine hohe Viskosität aufweist. Dieser Punkt kann erheblichen Einfluss auf die großtechnische Verarbeitbarkeit des Additives haben und wird im weiteren Verlauf dieser Arbeit näher betrachtet.


\subsection{Charakterisierung des Dosierstroms}

\subsection*{Bestimmung der Dichte nach }

\subsection*{Bestimmung der dynamischen Viskosität nach }
D

Rotationsviskosimeter

\subsubsection*{Bestimmung der Strömungsform}
Um den Dosierstrom des Verdickermittels entsprechend seiner Strömungseigenschaften charakterisieren zu können, wird zunächst mit der sogenannten \textsc{Reynoldszahl} die Strömungsform bestimmt. Sie ist eine dimensionslose Kennzahl und beschreibt das Verhältnis zwischen Tragheitskräften zu Reibungskräften in strömenden Flüssigkeiten und ist für durchströmte Rohrleitungen unter Gleichung \eqref{eq: reynolds} definiert. \cite{Foth.2014}

\begin{equation}
	\label{eq: reynolds}
	Re = \frac{d_H*\rho*\overline{u}}{\eta}
\end{equation}
\begin{parameter}
	Re 			& 	\textsc{Reynoldszahl} \\
	\eta 		& dynamische Viskosität des Fluids\\
	\rho 		& Dichte des Fluids\\
	d_H			&	hydraulischer Rohrdurchmesser\\
	\overline{u} & mittlere Strömungsgeschwindigkeit\\
\end{parameter}

Anhand der Reynoldszahl lässt sich nun mithilfe der Tabelle \ref{tab:stromung_reynolds}, die jeweilige Strömungsform zuordnen. Diese Zuordnung ist wichtig, da sich je nach Strömungsform unterschiedliche Einflussgrößen auf den Druckverlust ergeben. Beispielsweise hat für eine laminare Strömung die Wandrauigkeit der Leitung keinen Einfluss mehr, wohin gegen sie in turbulenten Strömungen maßgebliche Druckverluste hervorrufen kann. In laminaren Strömungen überwiegt hierbei der glättende Einfluss der Viskosität gegenüber den Rohrunebenheiten, während in turbulenten Strömungen weitere Wirbel erzeugt werden. \cite{Bschorer.2018}

% Table generated by Excel2LaTeX from sheet 'Daten'
\begin{table}[h!]
	\renewcommand*{\arraystretch}{1.2}
	\centering
	\rowcolors{2}{white}{gray!25}
	\caption{Strömungsformen und ihre Reynoldszahlen \cite{Foth.2014}}
	\label{tab:stromung_reynolds}
	%\resizebox{10.5cm}{!}{
		\begin{tabulary}{1.0\textwidth}{C|CCC}
			\hline
			\textbf{Strömungsform} & \textbf{Laminar} & \textbf{Übergangsbereich} & \textbf{Turbulent}\\
			\hline
			\textbf{Reynoldszahl} &	$< 2300$ & $2300$ bis $4000$& $>4000$\\
			\hline			
	\end{tabulary}
%}
\end{table}%
\FloatBarrier

\subsubsection*{Bestimmung des Druckverlustes}
Nach der Bestimmung der Reynoldszahl lässt sich nun mit Hilfe des \linebreak \textsc{Nikuradse-Colebrook-Moody}-Diagramms, nachfolgend \textsc{Moody}-Diagramm genannt, der Rohrreibungsbeiwert $\lambda$ bestimmen (siehe Abb. \ref{fig:moody}). Dieser Wert wiederum kann in Gleichung \eqref{eq:druckverlustbeiwert} eingesetzt werden um den Druckverlustbeiwert $\zeta_R$ für gerade Rohrleitungen zu bestimmen und lässt auf Basis der erweiterten \textsc{Bernoulli}-Gleichung \eqref{eq:druckverlust_zeta} den durch Reibung verursachten Druckverlust $\Delta p$ berechnen. \cite{Bschorer.2018}

\begin{equation}
	\label{eq:druckverlustbeiwert}
	\zeta_R = \lambda * \frac{L}{d}
\end{equation}
\begin{equation}
	\label{eq:druckverlust_zeta}
	\Delta p = \frac{1}{2}*\zeta_R*\rho*\overline{u}^2
\end{equation}
\begin{parameter}
	\zeta_R		& Druckverlustbeiwert für gerade Rohrstrecken\\
	L 			& Rohrleitungslänge\\
	d			& Rohrdurchmesser\\
	\Delta p	& Druckverlust \\
	\rho 			& Dichte des Fluids\\
	\overline{u} 	& mittlere Strömungsgeschwindigkeit\\
\end{parameter}


\begin{figure}[h!]
	\centering
	\includegraphics[width=1.0\textwidth]{img/R_Rohrreibungsbeiwert.jpg}
	\caption{\textsc{Nikuradse-Colebrook-Moody}-Diagramm \cite{Msimca.2017}}
	\label{fig:moody}
\end{figure}
\FloatBarrier
%Ende

\subsubsection*{Gesetz von  \textsc{Hagen}-\textsc{Poiseuille}}
Liegt für eine nicht-kompressible Flüssigkeit eine laminare Strömung vor, so ist es möglich die zuvor beschriebene Vorgehensweise zu vereinfachen und den auftretenden Druckverlust in einer geraden Rohrleitung direkt mit dem Gesetz von \textsc{Hagen}-\textsc{Poiseuille} zu bestimmen.  Der Druckverlust wird hierbei in Abhängigkeit vom Volumenstrom, der Rohrleitungslänge, des Rohrdurchmessers und der Viskosität berechnet. Die Definition des Gesetzes, aufgelöst nach dem Druckverlust, findet sich unter Gleichung \eqref{eq:hagen}. \cite{Foth.2005}

\begin{equation}
	\label{eq:hagen}
	\Delta p  = \frac{8*\eta*L*\dot{V}}{r^4*\pi}
\end{equation}
\begin{parameter}
	\Delta p	& Druckverlust \\
	\eta 		& dynamische Viskosität des Fluids\\
	\dot{V}		& Volumenstrom des Fluids\\
	r			& innerer Radius der Rohrleitung\\
	L 			& Rohrleitungslänge\\
\end{parameter}

Da das Gesetz von \textsc{Hagen}-\textsc{Poiseuille} bereits im \textsc{Moody}-Diagramm enthalten ist, können beide Vorgehensweisen genutzt werden um die jeweils andere Rechnung zu überprüfen. Sollen Rohrleitungseinbauten wie Armaturen, Ventile oder Bogenstücke einberechnet werden, vereinfacht jedoch aufgrund von tabellierten Druckverlustbeiwerten möglicher Einbauten die erweiterte \textsc{Bernoulli}-Gleichung die Berechnung des gesamten reibungsbedingten Druckverlustes.


\subsection{Dosierpumpen}
\subsubsection{oszillierende Verdrängerpumpen}

\subsubsection{rotierende Verdrängerpumpen}


%\subsection{Normen und Standards}
%
%
%\subsection{Densimeter}
%
%Norm für Viskositätsbestimmung
%
%https://www.din.de/de/neuer-inhalt/wdc-beuth:din21:306904236
%https://www.din.de/de/wdc-beuth:din21:512291
%https://www.din.de/de/neuer-inhalt/wdc-beuth:din21:329765890
%
%\subsection{Wirtschaftliche Aspekte und Entwicklungsperspektiven}