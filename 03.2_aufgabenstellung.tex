\section{Zielsetzung und Abgrenzung der Aufgabenstellung}
\label{sec:aufgabenstellung}
%eine Seite
%----------
%Motivation --> Ohrenöffner
%zentrale Frage
%Aufbau, Vorgehensweise, Methoden

\subsection{Zielsetzung der Aufgabenstellung}
Für die Herstellung der Acrylat-Copolymerdispersion AC 548 ist die Zugabe eines assoziativen Verdickermittels für die Alberdingk Boley Leuna GmbH als Prozessschritt nötig. Dieses Verdickermittel wird zugegeben, um die Viskosität der hergestellten, wässrigen Polymerlösung zu erhöhen und damit die rheologischen Eigenschaften für die Verwendung in Buntsteinputzen, Lacken und Farben sicherzustellen. Das derzeitig genutzte Verdickermittel macht es möglich, dass die Zugabe durch einen Abwiegeprozess und einem Fass als Dosierbehälter erfolgen kann. Diese Art der Dosierung erfordert jedoch eine gewisse Fließfähigkeit des zuzugebenen Mediums und die Akzeptanz einer kaum quantifizierten Dosierung. Weiterhin wird Personalzeit für Abwiege- und Transportprozesse benötigt.\\
Im Rahmen der Produktion soll diese Form der Dosierung technisch umgesetzt werden, um somit Personalzeit zu sparen, den Verbrauch an Verdickermittel gering zu halten und das Handling zu erleichtern. Aufgrund der Einstellung der Produktion für das derzeit verwendete Verdickermittel  Rheobyk-H 3300 VF der \mbox{\textsc{Byk-Chemie GmbH} }wird dieses in Zukunft durch TAFIGEL PUR 85 der \mbox{\textsc{Münzing Chemie Gmbh}} ersetzt. Dabei unterscheiden sich beide Verdickermittel grundlegend in ihrer Verarbeitbarkeit. Gerade das neu einzuführende Verdickermittel stellt durch seine hochviskosen Eigenschaften eine Herausforderung für den Dosierprozess dar.\\
Ziel dieser Arbeit ist es durch Charaktersierung des neuen Verdickermittels verschiedene Möglichkeiten der Dosierung zu recherchieren und zu diskutieren. Infolgedessen soll ein Konzept der Dosierung technisch geplant werden, welches den Forderungen des Unternehmens entspricht. Ausgewhälte Aspekte sind beispielsweise die Dosiergenauigkeit und das Handling durch die Produktion.

\subsection{Abgrenzung der Aufgabenstellung}
In Abgrenzung zur Aufgabenstellung wird in dieser Arbeit kein Bezug zur realen Umsetzung des vorgestellten Konzeptes genommen. Es werden lediglich Vorbetrachtungen und Möglichkeiten der Dosierung zusammengestellt und diskutiert. Diese Arbeit soll dabei ein gesamtheitliches Bild der Problematik zeichnen und mögliche Lösungen hierfür aufzeigen. Die Entscheidung ob das näherbeschriebene Konzept tatsächlich umgesetzt wird, bleibt an dieser Stelle dem betreuendem Unternehmen überlassen. Somit finden Aspekte der Inbetriebnahme oder Testreihen einer umgesetzten Dosierstation für diese Arbeit keine Bedeutung.