\section{Diskussion}
\label{sec:diskussion}
%Bewertung hinsichtlich:
%zentraler Aufgabe
%wissenschaftlicher, technischer Kontext
%Verallgemeinbarkeit
%persönliche Schlussfolgerungen

\subsection{Diskussion der experimentellen Untersuchungen}

\subsection{Diskussion der Entscheidungen}

\subsection{Diskussion der Rohrleitungsplanung}

\subsection{Diskussion der Gefährdungsbeurteilung}

Viskositätsmessung nach DIN EN ISO 2555, statt DIN EN ISO 3219 --> laut DIN EN ISO 2555 macht das Messverfahren für newtonsches Fluid keinen Unterschied

Trockenschrank könnte Lösungsmittel entfernen --> nicht signfikant --> dennoch geringere Viskosität durch Erwärmung

Becherglas 50 ml bei Verdünnungsmessung --> Anhaltspunkt für verlauf --> reicht für weitere Auswertung, da ungefähr selbes ergebnis bei 600 ml Becherglas

--> Dichte Somit könnten im betrachteten Temperaturbereich bereits Schwankungen von \SI{5}{\kelvin} einen registrierbaren Einfluss auf die Genauigkeit einer rein volumetrischen Dosierung haben.

Vergleich der Viskositäten in der Disskussion

Verdrängungsvolumen Wasser nicht 100\% Wirkungsgrad
Wirkungsgrad wahrscheinlich schon niedrigerer Drehzahl erschöpft