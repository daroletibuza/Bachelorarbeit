\section{Material und Methoden}
\label{sec:durchführung}
%Rahmenbedinungen der Untersuchung
%Auswahl, Einschränkungen und Begründungen
%Erhebungs-, Mess- und Auswertungsverfahren
%Womit und wie haben Sie untersucht ?

\subsection{Erfassung des Ist-Zustandes}
--> SAmmeln von Informationen im Werk

\subsection{Charakterisierung des Verdickungsmittels}

\subsubsection{Viskositätsmessungen}
--> nach DIN

\subsubsection{Verdünnungsverhalten}
--> Eigenregie

\subsubsection{Erwärmungsverhalten}

--> beim Hersteller angefragt

In der geplanten Dosierung soll auf das Verdickungsmittel TAFIGEL PUR 85 der \mbox{\textsc{Münzing Chemie Gmbh}} zurückgegriffen werden. Laut Hersteller handelt es sich hierbei um einen assoziativen Polyurethan-Verdicker, welcher durch Gerüstbildung zwischen Verdickermolekülen, Bindemittel und Pigmentpartikeln die gewünschte Viskosität hervorruft und stabilisiert. Diese Beschreibung deckt sich mit den vorangegangenen Beschreibung der Assoziativverdicker.\cite{MunzingChemieGmbH.2014}\\

\subsection{Charakterisierung der Anlage}

Kampagne
Mitarbeiterkosten sparen
Zeitersparnis
Einfachheit
Ex-Schutz
PLS
Prozesssicherheit

\subsection{Auswahl des Gebindes}
Preis

\subsection{Auswahl des Pumpentyps, sowie des Leitungsdurchmessers}
Gespräche geführt und Leitung hat sich nach möglichen Drücken in der Anlage und Pumpentyp, sowie Berechnungen gerichtet gerichtet