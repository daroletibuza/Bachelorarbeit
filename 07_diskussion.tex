\section{Diskussion der Ergebnisse}
\label{sec:diskussion}

Die Ergebnisse der Temperaturprofile deuten auf ein vielversprechendes Versuchsergebnis in Bezug auf Prozess 1. Betrachtet man die Messwertaufnahme der Temperaturprofile kann kritisiert werden, dass die Temperaturen extern und nicht über das Thermostat selbst gemessen wurden. Das kann dazuführen, dass Messwertabweichungen vorliegen können, diese werden jedoch als gering eingeschätzt. Entscheidender ist der Punkt der Solltemperatur für den Prozess und der Isttemperatur im Reaktor. Alle Temperaturrampen sind für die Temperatur des Reaktormantels mit der Temperierflüssigkeit, nicht jedoch auf eine sich im Reaktor befindliche Reaktionsmischung ausgelegt. Es ist demnach in der weiteren Bearbeitung der Prozessausführung zu beachten, dass möglicherweise zusätzliche Pufferzeiten für die Temperierung der Reaktionsmischung nötig sind. Grund dafür ist, dass nicht davon ausgegangen werden kann, dass die Temperatur der Reaktionsmischung direkt der Temperatur des Reaktormantels entspricht. Dennoch ist zu erwarten, dass die Pufferzeiten im Verhältnis zur Gesamtprozessdauer sehr gering ausfallen. Weiterhin ist zu beachten, dass falls für den Prozess ein anderes Thermostat oder ein anderer Kühlwasserstrom genutzt wird, die Heiz- und die Kühlkurve abweichend ausfallen können. Gerade in Bezug auf den Kühlwasserstrom ist es wahrscheinlich, dass wenn der Versuchsstand in einem anderen Labor aufgebaut wird, die Kühlzeit bei kühlerem Leistungswasser deutlich verringert werden kann.\\

Die Ergebnisse der Dosierung mittels Pumpen deuten im Vergleich zur Temperierung des Reaktors auf deutlich mehr Bearbeitungsbedarf bei weiterer Ausführung der Prozesse hin. Der Grund hierfür liegt darin, dass kein geregeltes Pumpensystem vorliegt und somit keiner der untersuchten Pumpen den geforderten Dosieranforderungen genügt. So können weder konstante \SI{135}{\milli \liter \per \hour} noch \SI{20}{\milli \liter \per \hour} des vereinfachten Prozesses erreicht werden (vgl. Tab. \ref{tab:verinfachteAnforderungen} und Abb. \ref{dia:bar_pumpe}). Der kleinste Volumenstrom für die geforderten Prozesse 1 und 2 mit \SI{6,78}{\milli \per \hour} schließt sich somit ebenfalls aus (vgl. Tab. \ref{tab:Anforderungen1_2}). Zwar scheint es laut Tab. \ref{dia:bar_pumpe} möglich, dass sowohl Zahnrradpumpe als auch Magnet-Membranpumpe den Feed 2 des Prozesses 1 mit \SI{500}{\milli \liter \per \hour} (siehe Tab. \ref{tab:Anforderungen1_2}) erfüllen könnten, jedoch müssen an dieser Stellen ebenfalls Einschränkungen getroffen werden. Die Magnet-Membranpumpe liegt in ihren zwei minimalen Einstellungen mit \SI{360}{\milli \liter \per \hour} und \SI{600}{\milli \liter \per \hour} zwischen dem geforderten Feed 2 mit \SI{500}{\milli \liter \per \hour}. Problem hierbei ist, dass die Magnetmembranpumpe nicht stufenlos geregelt werden kann und somit ein Volumenstrom direkt von der Pumpe aus nicht realisiert werden kann. Zu dem ist eine gute Reproduzierbarkeit in diesen Einstellungen laut Hersteller nicht gegeben \cite[S. 7]{prominent_beta_anleitung}. Im Vergleich dazu kann die genutzte Drehzahl der Zahnradpumpe, zwar stufenlos geregelt werden, jedoch ergibt sich für diese Pumpe das Problem der starken Abhängigkeit vom Füllstand des Behälters. Bestätigt ist die Abhängigkeit durch den linearen Zusammenhang zwischen der auftretenden Abweichung über die Zeit nach dem \textsc{Torricelli}-Theorem (siehe Abb. \ref{dia:pump_fehler}). Ein passender Volumenstrom müsste demnach regelmäßig über Drehzahl und Behälterfüllstand manuell nachjustiert werden.\\
Gute Reproduzierbarkeit und eine simple Handhabung sind damit jedoch nicht gegeben. Eine Möglichkeit zu große Volumenströme zu reduzieren könnte das Teilen des Volumenstromes sein, sowie das Verringern der Fläche der Austrittsöffnung für die Dosierung. Aber auch den Einfluss des Füllstandes könnte man mit einem Behälter mit vergrößerter Fläche des Flüssigkeitsspiegels in Betracht ziehen. Jede Möglichkeit müsste jedoch ebenfalls in einer weiteren Versuchsdurchführung untersucht werden. Eine elegantere Lösung könnte die Nutzung einer Dosierpumpe sein, die bereits den geforderten Ansprüchen entspricht, wie zum Beispiel die \linebreak \textsc{ProMinent Magnet-Membrandosierpumpe gamma/ X}. Über eine zusätzliche Sauglanze im Dosierbehälter wird die Höhe des Füllstandes an die Pumpe übertragen und die Leistung der Pumpe für die Dosierung entsprechend angepasst \cite{https:www.industr.com.16.06.2021, prominent_gamma}.

