\section{Ergebnisse}
\label{sec:ergebnisse}
%Was waren die Ergebnisse der Untersuchungen?

Verdickerviskosität

Berechnungen Druckverlust

Erwärmungsverhalten

Verdünnungsverhalten

Entscheidung für Pumpentyp und Leitungsdurchmesser mit Druckverlustrechnung 

Zeichnungen der Dosiervarianten

Zeichnugen im R\&I von der Umsetzung


\subsection{Ist-Zustand der Verdickungsmitteldosierung}
--> SAmmeln von Informationen im Werk

Kampagne
Mitarbeiterkosten sparen
Zeitersparnis
Einfachheit
Ex-Schutz
PLS
Prozesssicherheit

\subsubsection{Aktuelles Dosierverfahren}
\subsubsection{Produktionsweise}

\subsection{Eigenschaften des Verdickungsmittels}

\subsubsection{Viskositätsmessungen}
--> nach DIN

\subsubsection{Verdünnungsverhalten}
--> Eigenregie

\subsubsection{Erwärmungsverhalten}

--> beim Hersteller angefragt

In der geplanten Dosierung soll auf das Verdickungsmittel TAFIGEL PUR 85 der \mbox{\textsc{Münzing Chemie Gmbh}} zurückgegriffen werden. Laut Hersteller handelt es sich hierbei um einen assoziativen Polyurethan-Verdicker, welcher durch Gerüstbildung zwischen Verdickermolekülen, Bindemittel und Pigmentpartikeln die gewünschte Viskosität hervorruft und stabilisiert. Diese Beschreibung deckt sich mit den vorangegangenen Beschreibung der Assoziativverdicker.\cite{MunzingChemieGmbH.2014}\\


\subsection{Entscheidung für Gebindetyp}
Preis nachfrage
\subsubsection{Angebotsanfrage}
\subsubsection{Entscheidungsverfahren}

\subsection{Entscheidung für Pumpentyps und Leitungsdurchmessers}
\subsubsection{Berechnung des Druckverlustes}
\subsubsection{Literaturarbeit}
Bücher gelesen für Auswahlhilfe
\subsubsection{Fachgespräche und Angebotsanfrage}
\subsubsection{Pumpversuche}
\subsubsection{Entscheidungsverfahren}

\subsection{Entscheidung für Messverfahren}
\subsubsection{Literaturarbeit}
Bücher gelesen für Auswahlhilfe
\subsubsection{Fachgespräche und Angebotsanfrage}
\subsubsection{Entscheidungsverfahren}

\subsection{Technische Planung für Verdickungsmitteldosierung}
\subsubsection{R\&I- Fließbild der Verdickerdosierung}
\subsubsection{Theoretische Rohrleitungsplanung der Verdickerdosierung}
\subsubsection{Signalverarbeitungsplanung der Verdickerdosierung}

\subsection{Gefährdungsbeurteilung der geplanten Verdickmitteldosierung}


%\subsection{Auswahl des Pumpentyps, sowie des Leitungsdurchmessers}
%Gespräche geführt und Leitung hat sich nach möglichen Drücken in der Anlage und Pumpentyp, sowie Berechnungen gerichtet gerichtet