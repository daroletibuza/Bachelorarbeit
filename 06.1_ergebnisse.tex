\newpage
\section{Ergebnisse}
\label{sec:ergebnisse}
%Was waren die Ergebnisse der Untersuchungen?

\subsection{Verifizierung und Analyse des Dosierproblems}
\subsubsection{Ist-Zustand: Aktuelles Dosierverfahren und Produktionsweise}

\subsubsection{Soll-Zustand: Anforderung an die Dosierung}
%geringer Verschleiß
%leichte Wartung
%einfach zu Bedienen
%leichte Reinigung --> Spülbarkeit
%Dosiergenauigkeit und DOsierstrom
%--> SAmmeln von Informationen im Werk
%
%Kampagne
%Mitarbeiterkosten sparen
%Zeitersparnis
%Einfachheit
%Ex-Schutz
%PLS
%Prozesssicherheit
\subsubsection{Eigenschaften des Verdickungsmittels}
In der geplanten Dosierung soll auf das Verdickungsmittel TAFIGEL PUR 85 der \mbox{\textsc{Münzing Chemie Gmbh}} zurückgegriffen werden. Laut Hersteller handelt es sich hierbei um einen assoziativen Polyurethan-Verdicker, welcher durch Gerüstbildung zwischen Verdickermolekülen, Bindemittel und Pigmentpartikeln die gewünschte Viskosität hervorruft und stabilisiert. Diese Beschreibung deckt sich mit den vorangegangenen Beschreibung der Assoziativverdicker.\cite{MunzingChemieGmbH.2014}\\
--> Datenblätter

\subsubsection{Viskositätsmessungen}
--> nach DIN

\subsubsection{Verdünnungsverhalten}
--> Eigenregie

\subsubsection{Erwärmungsverhalten}

--> beim Hersteller angefragt

\subsubsection{Pumpversuche}

erster Volumenstrom erst nach einer Stunde zu erkennen 


\todo[inline]{Entscheidungsprobleme werden aufgrund des Umfangs und der schweren Nachvollziehbarkeit nicht dargestellt}

\subsection{Problem: Entscheidung für ein Verfahren}
%Verdünnen
%spülen
%erwrämen
%nichts 
%Rühren
%kombinationen

\subsection{Problem: Entscheidung für Gebindetyp}
%Preis nachfrage
%\subsubsection{Angebotsanfrage}
%\subsubsection{Entscheidungsverfahren}

\subsection{Problem: Direkte Zugabe mit Pumpe oder Dosierbehälter}

\subsection{Problem: Auswahl des Pumpentyps}
%\subsubsection{Literaturarbeit}
%Bücher gelesen für Auswahlhilfe
%\subsubsection{Fachgespräche und Angebotsanfrage}
%\subsubsection{Entscheidungsverfahren}

%\subsection{Auswahl des Pumpentyps, sowie des Leitungsdurchmessers}
%Gespräche geführt und Leitung hat sich nach möglichen Drücken in der Anlage und Pumpentyp, sowie Berechnungen gerichtet gerichtet

\subsection{Problem: Messverfahren}

\subsection{Technische Planung für Verdickungsmitteldosierung}
\subsubsection{Verfahrensfließbild der Verdickerdosierung}
\subsubsection{R\&I- Fließbild der Verdickerdosierung}
\subsubsection{Theoretische Rohrleitungsplanung der Verdickerdosierung}
%vertretbarer Druckverlust
%Rohrdurchmesser und Nenndruck
%--> Rohrleitungsklassifizierung

\subsubsection{Signalverarbeitungsplanung der Verdickerdosierung}

\subsection{Gefährdungsbeurteilung der geplanten Verdickungsmitteldosierung}


